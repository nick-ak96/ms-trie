\section{Related work} \label{c:relwork}

% intro to related work

Our work is related to storage of multisets and multiset containment queries. In this section we summarize previous work done 
for storage and retrieval of sets and multisets as well as work on (multi)set containment queries. Also, we discuss the applications 
of multiset-trie with respect to already existing data structures.


% Construction of multiset-trie


\subsection{Set-trie}
% set-trie: 
% - data structure that multiset-trie generalizes
% - stores a set of sets
% - supports set containment operations

The multiset-trie is closely related to the set-trie data structure proposed by Savnik~\cite{savnik2013index}. 
It is a trie-based data structure that is adapted to efficient storage and retrieval of sets. In particular, 
it stores a set of sets and supports set containment operations such as retrieval of the nearest sub- 
and supersets as well as retrieval of all sub- and supersets from the data structure.

The data structure multiset-trie generalizes the set-trie allowing to store a set of multisets. When the multiset-trie 
is restricted to store a set of sets the underlying data structure becomes a simple binary tree. Moreover, all the 
operations of the set-trie are also supported by the multiset-trie. The generalization is, of course, comes with a small penalty 
in performance if we compare the multiset-trie with the set-trie. The downside of such a generalization is that multiset-trie 
no longer supporting path compression that was obtained in set-trie. However, the design of multiset-trie provides a storage 
of multisets as well as constant worst-case time complexity of search operation independently of user input. 


\subsection{Inverted file}
% inverted index:
% - performance study of set containment queries (sequential signature files, signature trees, extendible signature hashing, inverted files)
% - performance comparison of multiset-trie to inverted index
% - containment query issues of inverted index

The performance study~\cite{Helmer2003} shows that the inverted file is the most efficient data structure for set containment queries 
among data structures: sequential signature files, signature trees, extendible signature hashing and inverted files. 
In our work we replicate the experiment with inverted file and compare the performance of set containment queries between inverted file and multiset-trie data structures. We user the same methodology as in related paper in order to implement containment queries for inverted file.
The results of this comparison can be found the Experiments section~\ref{s:exp5}.


% Application areas of multiset-trie


\subsection{Set containment joins}
[WIP]
% set containment joins:
% - review the paper: Set containment join revisited\cite{bouros2016set}
% - multiset containment join


\subsection{Multisets in relational databases}
[WIP]
% multisets in RDBMS and ORDBMS:
% - set-valued attributes
% - multiset-valued attributes
% - underlying datastructures for storage of set- and multiset-valued attributes


\subsection{Application of multiset-trie}
[WIP]
% application of multiset-trie
% - indexing of multisets
% - containment queries
% - multiset containment join
% - information retrieval








%
% Set theory, multiest theory, bag-of-words model
%\subsection{Multiset}
%Multiset is a widely used data structure in different areas of mathematics, physics 
%and computer science \cite{singh2007overview}. The theory of multisets is based 
%entirely on the theory of sets. However, classical mathematics does not deal with 
%multisets directly. Instead, one can define a multiset to be a \emph{family} of sets 
%or the functions on ordered pairs, where the members of a pair are an element and its 
%multiplicity. This means that mathematically the concepts of set such as cardinality, 
%set-containment operation, power set, equivalence classes and others are well defined 
%for multisets in terms of sets \cite{blizard1988multiset}. 
%
%The concept of a multiset can also be referred to the \emph{bag-of-words model}. 
%This model takes its origin from a linguistic context studied by Harris~\cite{harris1954distributional}. 
%According to the bag-of-words model, text can be represented as a bag (multiset) 
%of words, where an element is a word and the number of its occurrences in the 
%text is multiplicity. A bag of words does not keep track of grammar and ordering 
%of words. 

% Information retrieval
%\subsection{Information retrieval}
%\emph{Information retrieval (IR)} refers to a problem of finding material of an 
%unstructured nature that satisfies an information need~\cite{manning2008introduction}. 
%Usually, one is searching for a specific documents in a significantly large text 
%documents database. The size of a database makes the search a time consuming 
%operation. In order to resolve the issue IR systems pre-process data and create 
%indexes for future use in search operation. 
%
%The bag-of-words model is widely used in IR. In particular, such a representation of 
%text documents is used in database indexes when a full text search of a database 
%is required. The full-text search problem refers to indexing techniques for full-text 
%databases. The most efficient index nowadays uses the concept of an inverted index 
%\cite{zobel1992efficient}. 
%
%The proposed data structure multiset-trie can be used as an alternative implementation 
%of the search structure of an inverted index. It represents words as multisets and 
%stores them into data structure. The query processing is achieved using boolean 
%retrieval model \cite{manning2008introduction} and multiset containment operations. 
%Multiset containment operations of the multiset-trie implement the nearest neighbor 
%search queries which retrieve not only exact but also the most relevant results 
%to a user. 

%\subsection{Generalized search tree}
%The properties and operations of the multiset-trie makes it a competitor to the 
%most efficient implementation of a search tree the \emph{Generalized Search Tree 
%(GiST)}~\cite{broder2006indexing,hellerstein1995generalized,kornacker1999high} 
%that is used in inverted index. GiST is a very flexible data structure that can be 
%customized in order to behave like B+-tree, R-tree or RD-tree. It also provides 
%support for an extensible set of queries and data types that B+-tree, R-tree or RD-tree 
%do not support originally. GiST supports all the basic search tree operations such as 
%insert, delete and search, and in addition provides such extensions as the 
%nearest-neighbor search and multiset containment operations. The extensions 
%provided by GiST are native in the multiset-trie. The multiset-trie also has a fixed 
%height while GiST is a self-balanced tree and has to use additional methods in 
%order to preserve its balance. 