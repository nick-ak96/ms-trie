\section{Related work} \label{c:relwork}

% intro to related work
The data structure multiset-trie is related to the data structures and indexes designed to store and manage sets and multisets. We mainly focus on the related data structures and indexes that efficiently support the set and multiset containment queries. Firstly, we summarize our previous work on the data structure for managing sets in Section \ref{rel-strie}. Next, we present in Section \ref{rel-invfile} the related work on the inverted files, i.e., the index structure that serves as a central data structure in the area of Information Retrieval (abbr. IR) but also for storing sets and multisets in database management systems. The alternative to the inverted file is the signature tree that is presented in Section \ref{rel-signature}. Finally, we describe the related work in the area of database management systems in Section \ref{rel-dbms}. We review the novel index structures used for the containment queries and the proposed containment join algorithms. 

% Construction of multiset-trie

\subsection{Set-trie\label{rel-strie}}
% set-trie: 
% - data structure that multiset-trie generalizes
% - stores a set of sets
% - supports set containment operations

The multiset-trie is closely related to the set-trie data structure introduced by Savnik in~\cite{savnik2013index,savnik2021plos}. A set-trie is a trie data structure that is adapted for the efficient storage and retrieval of sets instead of the sequences of symbols. The set-trie provides the set containment operations such as retrieval of the \emph{nearest} subset or supersets as well as retrieval of \emph{all} subsets and supersets from the sets of sets.

Since we are storing sets where each element of the set can appear only once, and the ordering of elements is not important, the ordering of the elements from the alphabet can be used for guiding the search in set containment operations. Each set is represented in a set-trie by a path including the increasing elements of a set represented by set-trie nodes. Since all sets from a set-trie are ordered by the increasing value of the set elements, the children of each set-trie node $n$ can only be the elements larger than the element $n$. For a given set $s$ and a set-trie $S$, the set containment operations search solely the sub-tree of $S$ that includes all the sets (paths from a root to a set-trie node) that are the possible subsets or supersets of $s$.
%The size of such sub-tree depends on the size of $S$ and the shape of $s$.

The data structure multiset-trie generalizes the set-trie by providing storage for the set of multisets. When the multiset-trie is restricted to store a set of sets, the underlying data structure becomes a simple binary tree. Moreover, all the operations of the set-trie are also supported by the multiset-trie. The generalization comes with a small penalty in performance if we compare the multiset-trie with the set-trie in the performance of the set containment operations. The downside of such a generalization is that multiset-trie no longer supports path compression that was obtained in set-trie. However, the design of multiset-trie provides storage of multisets with constant worst-case time complexity of the set containment operations.

\subsection{Inverted file\label{rel-invfile}}
% inverted index:
% - performance study of set containment queries (sequential signature files, signature trees, extendible signature hashing, inverted files)
% - performance comparison of multiset-trie to inverted index
% - containment query issues of inverted index

The inverted file \cite{zobel1992efficient,zobel1998inverted,zobel2006inverted} is the most common data structure used to represent a collection of (multi)sets. In the area of IR \cite{manning2008introduction} the inverted files are used for searching documents that contain a given set of words. It is composed of two parts: a dictionary and the postings. The dictionary maps each word to a list of document identifiers together with the locations of words in documents. The dictionary is most often implemented by a variant of a search tree, such as a B+ tree. The postings are implemented as a list of positions that are stored on the disk because of the huge amount of documents usually indexed by the inverted file. Since we can have a large number of postings for one word, the postings are compressed. Furthermore, several possible optimizations exist in the representation and implementation of postings \cite{zobel2006inverted}, such as sorting of postings, a technique called skipping, and others.

The empirical analyses~\cite{Helmer2003,zobel1998inverted} show that the inverted file is the most efficient data structure for containment queries among the data structures: the sequential signature file, the signature tree, the extendible signature hashing, and the inverted file. 

% Application areas of multiset-trie

\subsection{Signature trees\label{rel-signature}}

A dynamically balanced signature tree \cite{Deppisch1986,Pfaltz1980}, or S-tree, is an alternative data structure for the representation of multisets. An S-tree stores objects on the basis of their attributes represented in the form of signatures. A signature of an object is formed by the discretization of object attributes. Each attribute is discretized by mapping the attribute values to a sequence of bits. The bit sequences are the abstractions of the values of object attributes. They are glued together to form a signature of an object. The mappings from attribute values to sequences of bits are defined in such a way that allows superimposing a set of signatures by a single signature. Such a signature is often formed by using the operation OR. This property of signatures provides the means for the construction of the hierarchy of signatures that is utilized for efficient search. The multisets can be effectively represented using signatures, and the superposition operation can be implemented by the operation OR. The use of the signature tree for the containment operations was studied by Tousidou et al. \cite{tousidou2002sigstruc}. They show that S-tree that uses linear hash partitioning can be used to implement the containment operations efficiently. 

\subsection{Multisets in relational databases\label{rel-dbms}}
% multisets in RDBMS and ORDBMS:
% - set-valued attributes
% - multiset-valued attributes
% - underlying datastructures for storage of set- and multiset-valued attributes

The index structures for the efficient implementation of the (multi)set containment queries were studied in the frame of the relational DBMS as well as the object-relational DBMS, where we can use multivalued attributes, including sets, multisets (bags), and lists. Zhang et al.\ \cite{Zhang2001} compared the performance of the containment queries implemented in a standard relational DBMS (abbr. RDBMS) to an Information retrieval engine. The results show that, in general, the IR engine performs better than an RDBMS on containment queries. They have identified the problems that reason for the poor performance of the containment operations in an RDBMS and showed that with some modifications, an RDBMS could perform this class of queries more efficiently. 

The joins in an object-relational DBMS can be defined by means of the containment operations. A number of \emph{containment join} methods have been proposed \cite{Ramasamy2000,Melnik2003,Jampani2005,Luo2015}. Ramasamy et al. propose the use of the partitioning set join that relies on the representation of sets by using signatures \cite{Ramasamy2000}. The signature-based representation allows efficient implementation of the set comparison operations. The partitioning set join was further improved by Melnik at al. \cite{Melnik2003} to handle large sets and to speed up the partitioning phase of the algorithm. Further, Jampani et al. introduce the PRETTI join algorithm that combines an inverted file with a prefix tree for the efficient implementation of the containment joins. The algorithm for $R\bowtie T$ recursively computes record identifiers from $T$ while traversing a prefix tree storing the sets from $R$. The algorithm uses a single intersection of two lists to enumerate the matching pairs of rid-s. The PRETTI algorithm was improved by Luo at al. \cite{Luo2015} by replacing the prefix tree with the Patricia tree. 
