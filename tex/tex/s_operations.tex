\section{Multiset-trie operations} \label{c:operations}

%
Let $\mathcal{M}$ be a multiset-trie and let $M$ be a set of multisets that are 
inserted into the multiset-trie $\mathcal{M}.$ We define a type \emph{Multiset} in 
order to use it as a representation of a multiset. The type \emph{Multiset} is 
an array $m$ of constant length $\sigma,$ where $i$-th cell represents the element 
$\phi^{-1}(i)$ from $\Sigma$ with multiplicity $m[i].$ From now on, we 
agree that the first cell of an array has index 1. Let us have an example of a 
\emph{Multiset} instance with $\sigma = 2:$
%
\begin{center}
\begin{tabular}{ccc}
Multiset & & Instance of type Multiset \\
$\{ 1,1,2 \}$ & $\cong $ & 
\begin{tabular}{|c|c|}
\hline 
2 & 1 \\
\hline 
\multicolumn{1}{c}{\tiny 1} & \multicolumn{1}{c}{\tiny 2} \\
\end{tabular}
\end{tabular}
\end{center}
%
The operations supported by the multiset-trie data structure are as
follows. 
%
\begin{enumerate}
\item \textsc{insert}($\mathcal{M}$, $m$): inserts a multiset $m$ into 
$\mathcal{M}$ if $m\not\in M;$
%
\item \textsc{search}($\mathcal{M}$, $m$): returns true if a multiset $m\in M$ 
for a given $\mathcal{M},$ and returns false otherwise;
%
\item \textsc{delete}($\mathcal{M}$, $m$): returns true if a multiset $m$ was 
successfully deleted from $\mathcal{M},$ and returns false otherwise (in case 
$m\not\in M$);
%
\item \textsc{submsetExistence}($\mathcal{M}$, $m$, $dev$): returns true if 
there exists a $x\in M$ for a given $\mathcal{M}$ such that $x\subseteq m$ 
and $| x[i] - m[i] |\leq dev$ for $1\leq i\leq \sigma$, and returns false otherwise; 
%
\item \textsc{supermsetExistence}($\mathcal{M}$, $m$, $dev$): returns true if 
there exists a $x\in M$ for a given $\mathcal{M}$ such that $x\supseteq m$ 
and $| x[i] - m[i] |\leq dev$ for $1\leq i\leq \sigma$, and returns false otherwise; 
%
\item \textsc{getAllSubmsets}($\mathcal{M}$, $m$, $dev$): returns the set of multisets 
$\{ x \in M : x\subseteq m \wedge |x[i]-m[i]|\leq dev \}$ for a given 
$\mathcal{M},$ where $1\leq i\leq \sigma;$
%
\item \textsc{getAllSupermsets}($\mathcal{M}$, $m$, $dev$): returns the set of multisets 
$\{ x\in M : x\supseteq m \wedge |x[i]-m[i]|\leq dev \}$ for a given $\mathcal{M},$
where $1\leq i\leq \sigma.$
%
\end{enumerate}

The parameter $dev$ is used to specify the maximal deviation in the multiplicity
of multiset containment operations. It is utilized to limit the search in multiset
containment queries to the sub-multisets and super-multisets that are
the closest to the input multiset $m$. In addition, we use $dev$ for
the implementation of multiset similarity search that, given $m$, retrieves
from $\mathcal{M}$ all sub-multisets or super-multisets that are similar to $m$
with respect to the deviation $dev$.

In the following subsections, we will present each operation of the multiset-trie 
data structure separately. 

Firstly we would like to describe some notations that will be used. The 
multiset-trie data structure is a recursive data structure. Hence, any subtree 
of a multiset-trie $\mathcal{M}$ is again a multiset-trie. This fact allows 
us to use the root node of a multiset-trie as its representative. Thus, the notation 
$\mathcal{M}$ will be used instead of $\mathcal{M}.root$ to refer to the root 
node of $\mathcal{M}.$ Non-existing or \emph{Null} nodes in multiset-trie will 
be marked as \emph{Null} and existing nodes at the level $LL$ will be marked 
as \emph{accepting} nodes. The array slicing operation will be used as follows. 
For a given array $a,$ $a[i:]$ represents the array obtained from $a$ by taking 
only the cells from index $i$ until the last cell. 


\subsection{Insert} \label{s:insert}
The procedure \textsc{insert}($\mathcal{M}$, $m$) inserts a new instance $m$ of 
type Multiset into multiset-trie $\mathcal{M}$. If the complete path already 
exists, then the procedure leaves the structure unchanged. Otherwise, it extends 
partially existing or creates a new complete path. The procedure does not return 
any result. The pseudocode for procedure \textsc{insert} is presented in 
Algorithm~\ref{alg:insert}.


\begin{algorithm}[h!]
\caption{Procedure \textsc{insert}}
\label{alg:insert}
\begin{algorithmic}[1]
\Procedure{\textsc{insert}}{$\mathcal{M}$, $m$}
\State $currentNode \gets \mathcal{M}$
\For{$i=1$ to $\sigma$}
\If {child $c_{m[i]}$ of $currentNode$ is \emph{Null}}
\State create new child $c_{m[i]}$ of $currentNode$
\EndIf
\State $currentNode \gets c_{m[i]}$
\EndFor
\State mark $currentNode$ as \emph{accepting}
\EndProcedure
\end{algorithmic}
\end{algorithm}

\subsection{Search}\label{s:search}
The function \textsc{search}($\mathcal{M}$, $m$) checks if the complete path corresponding to 
a given multiset $m$ exists in the structure $\mathcal{M}.$ The function returns 
true if the multiset $m$ exists in $\mathcal{M}$, and returns false otherwise. The 
function \textsc{search} is presented in Algorithm~\ref{alg:search}.

\begin{algorithm}[h!]
\caption{Function \textsc{search}}
\label{alg:search}
\begin{algorithmic}[1]
\Function{\textsc{search}}{$\mathcal{M}$, $m$}
\State $currentNode \gets \mathcal{M}$
\For{$i=1$ to $\sigma$}
\If {child $c_{m[i]}$ of $currentNode$ is \emph{Null}}
\State \Return False
\EndIf
\State $currentNode \gets c_{m[i]}$
\EndFor
\State \Return True
\EndFunction
\end{algorithmic}
\end{algorithm}

\subsection{Delete} \label{s:delete}
Function \textsc{delete}($\mathcal{M},$ $m$) searches for the complete path 
that corresponds to $m$ in order to remove it. If the path can not be found, the 
function immediately returns false. During the search, the function keeps track of the 
number of children for every node. It marks the nodes that have more than one child 
as \emph{parent nodes} and remembers the label of the child, which is a potential node 
where the sub-tree will be cut to remove the multiset. The parent node is needed to 
perform a removal because the multiset-trie is an explicit data structure. When the search 
is completed, the function removes the sub-tree of the last found parent node and 
returns true. In such a way, after deletion, all the prefixes for other multisets are 
preserved in $\mathcal{M}$ and $m$ is removed. The function \textsc{delete} is 
presented in Algorithm~\ref{alg:delete}.


\begin{algorithm}[h!]
\caption{Function \textsc{delete}}
\label{alg:delete}
\begin{algorithmic}[1]
\Function{\textsc{delete}}{$\mathcal{M},$ $m$}
\State $currentNode \gets \mathcal{M}$
\State $parent \gets currentNode$ 
\State $position \gets 1$
\For {$i=1$ to $\sigma$}
\If {child $c_{m[i]}$ of $currentNode$ is \emph{Null}}
\State \Return False
\EndIf
\State $numChildren \gets 0$
\For {$j=0$ to $n-1$}
\If {child $c_j$ of $currentNode$ is not \emph{Null}}
\State $numChildren\gets numChildren+1$
\EndIf
\EndFor
\If {$numChildren$ is not $1$}
\State $parent\gets currentNode$
\State $position \gets i$
\EndIf
\State $currentNode \gets c_{m[i]}$
\EndFor
\State child $c_{m[position]}$ of $parent\gets$ \emph{Null}
\State \Return True
\EndFunction
\end{algorithmic}
\end{algorithm}

\subsection{Sub-multiset and super-multiset existence}
\label{s:subexists} \label{s:superexists}
The functions \textsc{submsetExistence} and \textsc{supermsetExistence} are
symmetrical in the following sense. Let a multiset $m$ represent the borderline
in $\mathcal{M}$ defined by a path from the root to a leaf by following
the elements from $m$. The operation \textsc{submsetExistence} searches
the left part of $\mathcal{M}$ and the operation \textsc{supermsetExistence}
the right part of $\mathcal{M}$. 

The function \textsc{submsetExistence}($\mathcal{M},m,dev$) checks if there exists 
a multiset $x$ in $\mathcal{M},$ that satisfies the condition $x\subseteq m$ and 
$| x[i] - m[i] | \leq dev,$ where $1\leq i \leq \sigma.$ 
The function starts with searching for an exact match $x=m$ in $\mathcal{M},$ 
since $m\subseteq m$ by definition of sub-multiset inclusion. If an exact match is 
not found in $\mathcal{M},$ the function uses multiset-trie to find the closest 
(the largest) sub-multiset of $m$ in $\mathcal{M}$ by decreasing the multiplicity of 
elements in $m.$ The parameter $dev$ is used to limit a maximal deviation of 
multiplicity for a particular element in $x$ with respect to $m.$ 
At every level, the function tries to proceed with the largest possible multiplicity of 
an element that is provided by $m.$ However, when the function reaches some level 
where it meets a \emph{Null} node and can not go further using the path provided by 
$m,$ it decreases the multiplicity of an element 
that corresponds to a current level with respect to the specified maximal deviation. 
Thus, the function can decrease the multiplicity of an element or eventually skip it in 
order to find the closest $x\subseteq m.$ The function \textsc{submsetExistence} 
is presented in Algorithm~\ref{alg:subexists}.

\begin{algorithm}[h!]
\caption{Function \textsc{submsetExistence}}
\label{alg:subexists}
\begin{algorithmic}[1]
\Function{\textsc{submsetExistence}}{$\mathcal{M}, m, dev$}
\State $currentNode \gets \mathcal{M}$
\If {$currentNode$ is \emph{accepting}}
\State \Return True
\EndIf
\For {$i=m[1]$ down to max($0, m[1]-dev$)}
\If {child $c_i$ of $currentNode$ is not \emph{Null}}
\If {\textsc{submsetExistence}($c_i,m[2:], dev$)}
\State \Return True
\EndIf
\EndIf
\EndFor
\State \Return False
\EndFunction
\end{algorithmic}
\end{algorithm}

%\subsection{Super-multiset existence} \label{s:superexists}
The function \textsc{supermsetExistence}($\mathcal{M},m,dev$) checks if there 
exists super-multiset $x$ of a given multiset $m$ in $\mathcal{M},$ such that 
condition $| x[i] - m[i] |\leq dev$ is satisfied, where $1\leq i \leq \sigma.$
Symmetrically to the function \textsc{submsetExistence}, the function
\textsc{supermsetExistence} searches first for an exact match $x=m$ in
$\mathcal{M}.$ If such $x$ does not exist in $\mathcal{M},$ then
the function searches on the right side of the borderline (an exact match)
defined by $m$. Since we would like to find the closest (the smallest)
super-multiset we increase the multiplicity of elements in $m$ at every
level of $\mathcal(M)$ starting with the multiplicities of $m.$
The function \textsc{supermsetExistence} is presented in
Algorithm~\ref{alg:superxists}.


\begin{algorithm}[h!]
\caption{Function \textsc{supermsetExistence}}
\label{alg:superxists}
\begin{algorithmic}[1]
\Function{\textsc{supermsetExistence}}{$\mathcal{M},m,dev$}
\State $currentNode \gets \mathcal{M}$
\If {$currentNode$ is \emph{accepting} }
\State \Return True
\EndIf
\For {$i=m[1]$ to min($n-1, m[1] + dev$)}
\If {child $c_i$ of $currentNode$ is not \emph{Null}}
\If {\textsc{supermsetExistence}($c_i,$ $m[2:], dev$)}
\State \Return True
\EndIf
\EndIf
\EndFor
\State \Return False
\EndFunction
\end{algorithmic}
\end{algorithm}


\subsection{Get all sub-multisets and get all super-multisets} \label{s:getall}
The algorithms for functions \textsc{getAllSubmsets} and 
\textsc{getAllSupermsets} are based entirely on algorithms for 
\textsc{submsetExistence} and \textsc{supermsetExistence} functions that do not 
terminate on the first existing sub/super-multiset, but store the results and 
continue the procedure until all existing sub/super-multisets in $\mathcal{M}$ are 
found and stored. The functions \textsc{getAllSubmsets} and \textsc{getAllSupermsets} 
are presented in Algorithm~\ref{alg:getallsub} and Algorithm~\ref{alg:getallsup} respectively.

In order to record a multiset during multiset-trie traversal, we use the variable $x$ in the algorithms.
It is an empty array of size $\sigma$ where we store multiplicities of elements at 
each level as we traverse the tree. The variable $result$ is used as a container for storing 
sub-multisets of $m$ found during traversal. Both variables $x$ and $result$ are presented 
as global, however, they could be passed to the recursive function as parameters.

\begin{algorithm}[h!]
\caption{Function \textsc{getAllSubmsets}}
\label{alg:getallsub}
\begin{algorithmic}[1]
\State $result \gets$ empty container
\State $x \gets$ empty array of size $\sigma$
\Function{\textsc{getAllSubmsets}}{$\mathcal{M}, m, dev$}
\State $currentNode \gets \mathcal{M}$
\If {$currentNode$ is \emph{accepting}}
\State add copy of $x$ to $result$ 
\EndIf
\For {$i=m[1]$ down to max($0, m[1]-dev$)}
\If {child $c_i$ of $currentNode$ is not \emph{Null}}
\State $x[1]\gets i$
\State \textsc{getAllSubmsets}($c_i,m[2:], dev$)
\EndIf
\EndFor
\EndFunction
\end{algorithmic}
\end{algorithm}


\begin{algorithm}[h!]
\caption{Function \textsc{getAllSupermsets}}
\label{alg:getallsup}
\begin{algorithmic}[1]
\State $result \gets$ empty container
\State $x \gets$ empty array of size $\sigma$
\Function{\textsc{getAllSupermsets}}{$\mathcal{M},m,dev$}
\State $currentNode \gets \mathcal{M}$
\If {$currentNode$ is \emph{accepting} }
\State add copy of $x$ to $result$ 
\EndIf
\For {$i=m[1]$ to min($n-1, m[1] + dev$)}
\If {child $c_i$ of $currentNode$ is not \emph{Null}}
\State $x[1]\gets i$
\textsc{getAllSupermsets}($c_i,$ $m[2:], dev$)
\EndIf
\EndFor
\EndFunction
\end{algorithmic}
\end{algorithm}