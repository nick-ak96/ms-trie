\pagenumbering{Roman}
\pagestyle{empty}
\begin{center}
\noindent \large UNIVERZA NA PRIMORSKEM\\
\large FAKULTETA ZA MATEMATIKO, NARAVOSLOVJE IN\\
INFORMACIJSKE TEHNOLOGIJE


\normalsize
\vspace{5.5cm}
Zaklju\v cna naloga\\
(Final project paper)\\
\textbf{\large Predstavitev objektov z multimno\v zicami v sistemih za informacijsko povpra\v sevanje}\\
\normalsize
(Multiset representation of objects in information retrieval systems)\\
\end{center}

\begin{flushleft}
\vspace{5cm}
\noindent Ime in priimek: Mikita Akulich
% add your first name and last name in the line above
\\
\noindent \v Studijski program: Matematika
% add your study program in the line above
\\
\noindent Mentor: doc. dr. Iztok Savnik
% add the academic title, first name, and last name of your mentor in the line above
\\
\noindent Somentor: doc. dr. Matja\v z Krnc
% if you have a co-mentor, add his/her academic title, first name, and last name in the line above
% if you do not have a co-mentor, delete the line above and the line below
\\
\end{flushleft}

\vspace{4cm}
\begin{center}
\large \textbf{Koper, september 2017}
% add the month and year of submission of your final project paper
\end{center}
\newpage

\pagestyle{fancy}
%%%%%%%%%%%%%%%%%%%%%%%%%%%%%%% Key words documentation (Slovene and English) %%%%%%%%%%%

\section*{Klju\v cna dokumentacijska informacija}

\medskip
\begin{center}
\fbox{\parbox{\linewidth}{
\vspace{0.2cm}
\noindent
Ime in PRIIMEK: Mikita AKULICH \vspace{0.5cm}\\
Naslov zaklju\v cne naloge: Predstavitev objektov z multimno\v zicami v sistemih za informacijsko povpra\v sevanje \vspace{0.5cm}\\
Kraj: Koper \vspace{0.5cm}\\
Leto: 2017 \vspace{0.5cm}\\
\v Stevilo listov: 40\hspace{2cm} \v Stevilo slik: 15\hspace{2.6cm} \\
\v Stevilo referenc: 14 \vspace{0.5cm}\\
Mentor: doc. dr. Iztok Savnik \vspace{0.5cm}\\
Somentor: doc. dr. Matja\v z Krnc \vspace{0.5cm}\\
Klju\v cne besede: multimno\v zica, vre\v ca-besed, trie, multiset-trie, informacijsko povpra\v sevanje, obrnjen indeks, full-text search \vspace{0.5cm} \\
Math.~Subj.~Class.~(2010): 68P05, 68P20, 68Q87 \vspace{0.5cm}\\
{\bf Izvle\v cek:}
\vspace{0.2cm} \\
V zaklju\v cni nalogi je predstavljena multiset-trie, nova podatkovna struktura, ki deluje na objektih predstavljenih 
z multimno\v zicami. 
Multiset-trie je podatkovna struktura, ki temelji na iskalnem drevesu in ima podobne lastnosti kot trie. 
Vklju\v cuje vse standardne operacije iskalnega drevesa skupaj z operacijami vsebovanja nad multimno\v zicami. 
Operacije vsebovanja, ki so definirane z multiset-trie, so izpeljane iz operacij pod-multimno\v zica in nad-multimno\v zica. 
Te operacije se uporabljajo za izvajanje razli\v cnih poizvedb, ki delujejo na multimno\v zicah v multiset-trie. 
Ena izmed najpomembnej\v sih poizvedb je iskanje najbli\v zjega soseda glede na vhodno multimno\v zico. 
Iskanje najbli\v zjega soseda v multiset-trie je dobra alternativa operacijam nad indeksnimi strukturami, ki se uporabljajo v sistemih 
za informacijsko povpra\v sevanje.
Na\v sa raziskava je osredoto\v cena na uporabo multiset-trie v sistemih za iskanje po celotnem besedilu.
}}
\end{center}

\newpage

\section*{Key words documentation}

\medskip

\begin{center}
\fbox{\parbox{\linewidth}{
\vspace{0.2cm}
\noindent
Name and SURNAME: Mikita AKULICH \vspace{0.5cm}\\
Title of final project paper: Multiset representation of objects in information retrieval systems \vspace{0.5cm}\\
Place: Koper \vspace{0.5cm}\\
Year: 2017 \vspace{0.5cm}\\
Number of pages: 40\hspace{1.6cm} Number of figures: 15\\
Number of references: 14 \vspace{0.5cm}\\
Mentor: Assist.~Prof.~Iztok~Savnik, PhD\vspace{0.5cm}\\
Co-Mentor: Assist.~Prof.~Matja\v z~Krnc, PhD\vspace{0.5cm}\\
Keywords: multiset, bag-of-words, trie, multiset-trie, information retrieval, inverted index, full-text search \vspace{0.5cm}\\
Math.~Subj.~Class.~(2010): 68P05, 68P20, 68Q87 \vspace{0.5cm}\\
{\bf Abstract:}
\vspace{0.2cm} \\
In this thesis we will present the multiset-trie, a new data structure that operates 
on objects represented as multisets. The multiset-trie is a search-tree-based 
data structure with the properties similar to those of a trie. It implements all standard search 
tree operations together with the special multiset containment operations. Multiset 
containment operations supported by the multiset-trie are submultiset and supermultiset. 
These operations are used for implementation of different queries that can be performed 
on multisets in a multiset-trie. One of the most important queries is the search of 
the nearest neighbor given an input object. The nearest neighbor search of a 
multiset-trie makes it a good alternative for the index data structures that are 
used in information retrieval systems. In particular, our research is focused on the 
application of the multiset-trie to full-text search systems. 
}}
\end{center}




%%%%%%%%%%%%%%%%%%%%%%%%%%%%%%% Acknowledgement %%%%%%%%%%%%%%%%%%%%%%%%%%%%%%%%%%%%%

\newpage
\section*{Acknowledgement}

I would like to thank my mentor assist. prof. Iztok Savnik and co-mentor 
assist. prof. Matja\v z Krnc for introducing me to this topic and their guidance 
for completing my final thesis. I would also like to thank assoc. prof. Riste 
\v Skrekovski for organizing a lecture at Faculty of Mathematics and Physics at 
University of Ljubljana where I had an opportunity to present our project and 
receive a practical feedback. Finally, I would like to thank Faculty of Mathematics, 
Natural Sciences and Information Technologies for given support trough supporting me 
with a scholarship during my stay, as well as the financial help for attending competitions 
and conferences.

%%%%%%%%%%%%%%%%%%%%%%%%%%%%% Table of contents, list of figures, etc. %%%%%%%%%%%%%%%%%%%%%%%%%%%%%%
\newpage

\tableofcontents
\addtocontents{toc}{\protect\thispagestyle{fancy}}
% if there are no tables in your final project paper, delete the following three lines
%\newpage
%\listoftables
%\addtocontents{lot}{\protect\thispagestyle{fancy}}
% if there are no figures in your final project paper, delete the following three lines
\newpage
\listoffigures
\addtocontents{lof}{\protect\thispagestyle{fancy}}
\newpage
% Since the appendices are not numbered, we also do not want to show the dots to their (non-existing) page numbers.
%\renewcommand{\cftdot}{}
%\listofappendices
%\thispagestyle{fancy}
%\newpage

\normalsize