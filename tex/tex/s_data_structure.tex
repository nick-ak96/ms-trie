\section{Multiset-trie data structure} \label{c:description}
%
Let $\Sigma$ be a set of distinct symbols that define an alphabet and let 
$\sigma$ be the cardinality of $\Sigma.$ The \emph{multiset-trie} data structure 
stores multisets that are composed of symbols from the alphabet $\Sigma.$ It 
provides the basic tree data structure operations such as insert, delete and 
search together with multiset containment and membership operations such as 
submultiset and supermultiset that will be discussed in the next section 
in greater details.

Multiset ignores the ordering of its elements by definition, which allows us to 
define a bijective mapping $\phi:\Sigma \rightarrow I,$ where $I$ is the set of 
integers $\{ 1,2,3, \ldots, \sigma \}.$ In this way, we obtain an indexing of 
elements from the alphabet $\Sigma,$ so we can work directly with integers 
rather then with specific symbols from $\Sigma.$

The multiset-trie is an $n$-ary tree based data structure with the properties of 
trie. A node in multiset-trie always has degree $n,$ i.e. $n$ children. Some of 
the children may be \emph{Null} (non-existing), but the number of \emph{Null} 
children can be at most $n-1.$ All the children of a node, including the \emph{Null} 
children, are labeled from left to right with labels $c_j,$ where $j\in \{ 0, 1, \ldots, n-1 \}.$ 
Every two child nodes $u$ and $v$ that share the same parent node have different labels.

%The multiset-trie is an $n$-ary tree based structure. A node can have a degree 
%from $1$ to $n,$ depending on the number of existing children. The edges from 
%parent node to children have labels $c_j,$ where $j\in \{ 0, 1, \ldots, n-1 \}.$ 
%The edges are labeled from left to right with the condition that there are no 
%two edges with the same label coming from a parent node. 

Nodes that have equal height in a multiset-trie form a level. The height of a 
multiset-trie is always $\sigma+1$ if at least one multiset is in structure. 
The height of the root node (the first level) is defined to be 1.
%
Levels in multiset-trie are enumerated by their height, i.e. a level $L_i$ has 
height $i.$ The connection between level height in a multiset-trie and symbols 
from alphabet $\Sigma$ is defined as follows. A level $L_i,$ where 
$i\in\{ 1,2,\ldots, \sigma \}$ represents a symbol $s\in\Sigma,$ such that 
$\phi^{-1}(i) = s.$ The last level $L_{\sigma+1}$ does not represent any symbol 
and is named \emph{leaf level} ($LL$ for short).

Since every level, except $LL$ represents a symbol from $\Sigma,$ we can define 
a transition between nodes that are located at different levels in a multiset-trie. 
%
Consider two nodes $u,v$ in a multiset-trie at levels $L_i, L_{i+1}$ respectively, 
where $i\in\{1,2,\ldots,\sigma\}.$ Let a node $u$ be a parent node of a node $v$ 
and consequently a node $v$ be a child node of a node $u.$ Suppose that a child 
node $v$ is not \emph{Null} and has a label $c_j,$ where $j\in\{ 0,1,\ldots, n-1 \}.$ 
%
Then the \emph{path} $u\rightarrow v$ represents a symbol $s\in\Sigma$ with 
multiplicity $j,$ such that $\phi^{-1}(i) = s.$ 
%
Such a transition $u\rightarrow v$ is called a \emph{path of length} $1$ and is 
allowed if and only if a node $v$ is not \emph{Null} and $u$ is a parent node of 
a node $v.$ If a node $v$ has label $c_0,$ then the path $u\rightarrow v$ 
represents a symbol with the multiplicity $0$ respectively i.e. an empty symbol.

We define a \emph{complete path} to be the path of length $\sigma$ in a 
multiset-trie with the end points at root node (the 1st level) and $LL$. Thus, 
a multiset $m$ is inserted into a multiset-trie if and only if there exists a 
complete path in a multiset-trie that corresponds to $m.$
%
Note that every complete path in a multiset-trie is unique. Therefore, the multisets 
that share a common prefix in a multiset-trie can have a common path of length at 
most $\sigma-1.$ The complete path that passes through nodes labeled by $c_0$ 
on all levels represents an empty multiset or an empty set.
%
Thus, any multiset $m$ that is composed of symbols from $\Sigma$ with maximum 
multiplicity not greater than $n-1$ can be represented by a complete path in a multiset-trie.

Let us have an example of a multiset-trie data structure. Let $\sigma = 2$ and 
$\Sigma = I = \{ 1,2 \}$ respectively, so the mapping $\phi$ is an identity mapping. 
Fix the degree of a node $n=3,$ so the maximal multiplicity of an element in 
a multiset is $n-1=2.$ The figure~\ref{fig:sketch} presents the multiset-trie that 
contains multisets $\emptyset, \{ 1,1,2 \},$ $\{ 1,2,2 \},$ $\{ 2 \},$ $\{ 1,2 \},$ $\{ 2,2 \}.$ 
The \emph{Null} children are omitted on the figure.

\begin{figure}[h!]
\centering
\begin{tikzpicture}[>=stealth',
level 1/.style={sibling distance = 2.5cm},
level 2/.style={sibling distance = 1cm},
level distance = 1.5cm]
\node (R) [node1] {\tiny Root}
	child { node [node1] {} 
		child { node [node1] {} edge from parent node [near end, left] {$c_0$}}
		child { node [node1] {} edge from parent node [near end, left] {$c_1$}}
		child { node [node1] {} edge from parent node [near end, right] {$c_2$}}
		edge from parent node [near end, left] {$c_0$}	
	}
	child { node [node1] {}
		child { node [node1] {} edge from parent node [near end, left] {$c_1$}}
		child { node [node1] {} edge from parent node [near end, right] {$c_2$}}
		edge from parent node [near end, left] {$c_1$}
	}
	child { node [node1] {} 
		child { node [node1] {} edge from parent node [near end, left] {$c_1$}}
		edge from parent node [near end, right] {$c_2$} 
};	
%----------------- Level labels --------------	
	\begin{scope}[every node/.style={right}]
		\path (R 				-| R-3-1) ++(5mm,0) node {$L_1$};
		\path (R-3 			-| R-3-1) ++(5mm,0) node {$L_2$};
		\path (R-3-1		-| R-3-1) ++(5mm,0) node {$LL$};
	\end{scope}			
\end{tikzpicture}
\caption{Example of multiset-trie structure.}
\label{fig:sketch}
\end{figure}

Let a pair $(L_i, c_j)$ represents a node with label $c_j$ at a level $L_i.$ 
The pair $(L_1,c_j)$ is equivalent to $(L_1,root),$ since the first level has 
the root node only. According to the figure~\ref{fig:sketch} we can extract 
inserted multisets as follows:

\begin{eqnarray*}
(L_1,root) \rightarrow (L_2,c_0) \rightarrow (LL,c_0) & = & \{ 1^0, 2^0 \} = \emptyset \\
(L_1,root) \rightarrow (L_2,c_0) \rightarrow (LL,c_1) & = & \{ 1^0, 2^1 \} = \{ 2 \} \\
(L_1,root) \rightarrow (L_2,c_0) \rightarrow (LL,c_2) & = & \{ 1^0, 2^2 \} = \{ 2,2 \} \\
(L_1,root) \rightarrow (L_2,c_1) \rightarrow (LL,c_1) & = & \{ 1^1, 2^1 \} = \{ 1,2 \} \\
(L_1,root) \rightarrow (L_2,c_1) \rightarrow (LL,c_2) & = & \{ 1^1, 2^2 \} = \{ 1,2,2 \} \\
(L_1,root) \rightarrow (L_2,c_2) \rightarrow (LL,c_1) & = & \{ 1^2, 2^1 \} = \{ 1,1,2 \} 
\end{eqnarray*}
where $e^k$ represents an element $e$ with multiplicity $k.$