\section{Introduction} \label{c:introduction}
[Will be rewritten]
%
% some intro
% data structures for indexing databases (multivalued attributes)
% look at set-trie (data-mining [hypotheses that have been used already ], etc)
% structure can be used for sets
% use of submultiset operations
% index for text search (ordering is not preserved) 
% set-trie reference

During recent years the popularity of digital data has increased. All sorts of 
information such as text, audio and video can now be accessed by searching 
\emph{information retrieval systems}. Information retrieval (IR) is the science of 
searching information units in a collection of data. Most commonly IR systems 
are used for searching a text-based content such as text documents in a database. 
Such IR systems are called \emph{full-text search systems}. 

Full-text search techniques can be applied directly on the database. However, 
it is a very expensive procedure in terms of running time complexity, because it 
requires a frequent accesses to the database. In order to reduce this number, indexes 
were invented. Indexes narrow down the search using pre-generated meta data 
constructed from the data in the database. Furthermore the meta data can be 
organized in a data structure that would provide fast retrieval of data according 
to search queries.

In IR most systems use the concept of an inverted index to achieve full-text 
indexing of a database. Inverted index consists of two parts: \emph{postings} 
and \emph{dictionary}, a search structure that is used to locate a specific entries 
in a posting. A posting entry can be created on different levels depending on data 
that needs to be indexed. Most common for full-text indexes are document level 
entries. In this context a posting is defined to be a list of identifiers or keys that are 
further used to locate a specific document in the database \cite{zobel1992efficient}.

An index is a search structure that is used to process user queries. The query 
can be processed in different ways according to the \emph{retrieval model}. The 
retrieval model that will be discussed in the thesis is the \emph{boolean retrieval 
model} that views each document in a database as a set of words. The document 
itself is an information unit a retrieval system is built over. In our case an 
information unit is defined to be a textual document. 

The boolean retrieval model is based on set and multiset theory together with 
boolean algebra. \emph{Set and multiset containment operations} are used to 
derive the similarities between objects, and consequently make decisions on their 
association. Thus, the multiset containment operations allow us to 
search objects not only with exact queries but also to retrieve the most relevant 
set of results that satisfy a given search query \cite{baeza1999modern,manning2008introduction,zobel2006inverted}.

% research topic
% walking through the new terms and so on
% brief overview of the literature (relations to other work etc.)
%

The dictionary (search structure) of the inverted index can be organized in different ways in 
order to meet the required types of queries and specification of data. For example, 
it can be organized as a search tree, hash map, array, heap, linked list, etc. In our 
research we will be focused on search trees. The most efficient search tree index 
nowadays is the Generalized Search Tree (GiST). Its flexibility stems from combining 
functionality of B+-trees, R-trees and RD-trees. GiST further extends their functionality 
providing support for a variety of data types together with the nearest-neighbor 
search \cite{broder2006indexing,hellerstein1995generalized,kornacker1999high}.

% description of innovations
%
The proposed data structure \emph{multiset-trie} can be used as an alternative 
implementation of the search structure in an inverted index. It is an extension of 
the \emph{set-trie} data structure proposed by Savnik \cite{savnik2013index}. 
Set-trie is a trie based data structure that is used for storing and fast retrieval of 
objects represented as sets. The set-trie provides the nearest-neighbor search by 
implementing methods that perform set-containment queries. Multiset-trie extends 
the abilities of set-trie and provides support for storing and retrieving objects 
that can be represented as both sets and multisets. It also implements 
multiset-containment methods together with the basic tree methods such as search, 
deletion and insertion.

The multiset-trie is an $n$-ary tree based data structure with properties similar to 
those of a trie. 
This particular combination allows us to associate multisets with a collection 
of nodes in a tree. Every node represents a symbol with particular multiplicity. 
%
Multiset-trie is a kind of search tree. Similarly to a trie, it uses common prefixes 
to narrow down the search. Unlike the compact prefix tree, Patricia, the multiset-trie 
does not provide the ability to compress a path. However, the absence of path 
compression makes the multiset-trie a perfectly height-balanced tree.

The multiset-trie is designed for efficient execution of the multiset containment 
operations. In particular, it supports the operations \textsc{submsetExistence}, 
\textsc{supermsetExistence}, \textsc{getAllSubmsets} and \textsc{getAllSupermsets}. 
The so-called "existence" queries implement the nearest-neighbor search queries. The 
functions \textsc{submsetExistence} and \textsc{supermsetExistence} search for 
the closest submultiset and supermultiset in the multiset-trie respectively and return 
an answer whether such a multiset exists in the data structure. The so-called 
"getAll" functions act in the same way as "existence" functions, but they do not 
terminate once they have reached the desired multiset. Alternatively, these functions 
store the results and continue until all the multisets that satisfy the query are retrieved. 

% paper organization
%
Let us now present the organization of the thesis. 
In the following Chapter~\ref{c:description} we present the description of the 
multiset-trie data structure. The representation of multisets in multiset-trie is 
explained in detail. The organization of the data structure is also presented 
graphically.
%
In Chapter~\ref{c:operations} we present operations that multiset-trie currently 
supports. The multiset containment functions \textsc{submsetExistence}, 
\textsc{supermsetExistence}, \textsc{getAllSubmsets} and \textsc{getAllSupermsets} 
are presented together with the basic search tree functions such as \textsc{insert}, 
\textsc{delete} and \textsc{search}. The algorithms in pseudo code are presented 
as well. 
%
The description of multiset-trie functions and procedures is followed by the 
mathematical analysis of their complexity in Chapter~\ref{c:analysis}. 
In this analysis, we make an assumption that multisets are constructed uniformly at 
random and are parametrized by several parameters, such as multiplicity and the alphabet 
$\Sigma.$ By using probabilistic tools we describe time complexity of the algorithms 
and space complexity of the structure.
%
Further, in Chapter~\ref{c:experiments} we present an empirical study of the 
multiset-trie. Artificially generated as well as real-world data sets are used in experiments. 
The experiments are dedicated to testing the performance of the data structure while 
varying selected parameters. The experiments also show some methods for 
optimizing a multiset-trie.
%
The Chapter~\ref{c:relwork} presents related work. The connection to the set-trie 
data structure \cite{savnik2013index} is discussed more explicitly. We also relate 
the multiset-trie to the information retrieval systems. In particular, we refer to the 
inverted index data structure and discuss how the multiset-trie can be used as a 
database index.
%
Finally, in Chapter~\ref{c:conclusions} our conclusion about the multiset-trie 
data structure and discussion of future work are presented.