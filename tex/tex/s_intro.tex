\section{Introduction} \label{c:introduction}
%
% general intro into problem
%
A multiset is a data structure that represents a collection of elements. It generalizes a set data structure 
by allowing duplicate elements in a collection. 
%
Data with duplicate elements is encountered in a wide variety of domains and applications. For example, 
in databases or data mining. In particular, processing of multisets has attracted significant attention in 
recent years. 
%
There are discussions about storage and representation of multi-valued attributes in databases, organization 
of data with shared content, containment joins, efficient algorithms for containment queries, compression 
of multisets, etc~\cite{bouros2016set,gripon2012compressing,ross2004symmetric,steinruecken2015compressing}.

Multiset data structure is usually implemented as a search tree and its elements can be accessed by a key. 
%
This approach is good when individual elements of a multiset are of interest. However, it is not as optimal for operations 
that use containment relation between multisets, also known as containment queries. 
%
Containment relation allows us to operate directly with the collections that are associated with a multiset such as a set 
of submultiset and a set of supermultisets.
%
Currently, the solution to searching or retrieval of a set of multisets using containment queries involves inverted file, 
signature tree or B+ tree~\cite{Helmer2003}. The solution provides a key-value look-up for elements of a multiset implying that 
containment operations are still element-wise.

% --------------------------------------------------------------------------------------------------------------------------------------------------------------------

% description of innovations
%
In this paper we propose a data structure \emph{multiset-trie} that is designed for storage and 
processing of a set of multisets. It extends the \emph{set-trie} data structure proposed by 
Savnik~\cite{savnik2013index} that was designed for storage and processing of a set of sets. 
%
Set-trie is a trie based data structure that provides a fast search and retrieval of sets and implements 
set-containment operations. 
%
Multiset-trie generalizes set-trie and can work with a set of sets (as set-trie) or with 
a set of multisets providing efficient multiset-containment operations.

% intro into mstrie
The multiset-trie is an $n$-ary tree based data structure with properties similar to those of a trie. 
Multisets are associated with a collection of nodes in a tree such that every node represents an element 
of a multiset with particular multiplicity. 
%
Multiset-trie is a kind of search tree. Similarly to a trie, it uses common prefixes for shared data representation.
Unlike the compact prefix tree, the multiset-trie does not support path compression. However, the absence of path 
compression makes the multiset-trie a perfectly height-balanced tree.
%
The multiset-trie handles multisets directly by having access to each of the element 
without the need to reconstruct them for processing, which allows fast retrieval and 
containment operations. In particular, it supports submultiset and supermultiset queries.
The operations allow to find and retrieve the closest submultisets or supermultisets 
as well as to find and retrieve all of them.

% --------------------------------------------------------------------------------------------------------------------------------------------------------------------

% key results
Empirical studies on real data show, that multiset-trie is sensitive to the context, which can 
be further used to optimize data structure for particular data. Moreover, knowing the data 
it is possible to estimate the overall performance of the multiset-trie both time and space 
related using our mathematical theory that describes multiset-trie.
%
The comparative study shows how efficient multiset-trie is by outperforming inverted file in 
both equality and containment queries by up to 4 orders of magnitude in the time consumed by 
query. 

% --------------------------------------------------------------------------------------------------------------------------------------------------------------------

% paper organization
%
In the following Section~\ref{c:description} we present the organization of 
multiset-trie data structure in detail.
%
Next, in Section~\ref{c:operations} we present operations that multiset-trie currently 
supports. This includes multiset containment and the basic search tree functions. 
The algorithms in pseudo code are presented as well. 
%
The description of multiset-trie functions and procedures is followed by the 
mathematical analysis of their complexity in Section~\ref{c:analysis}. 
The main assumption is that multisets are constructed uniformly at random 
with bounded cardinality. By using probabilistic tools we describe time complexity of 
the algorithms and space complexity of the structure.
%
Further, in Section~\ref{c:experiments} we present an empirical study of the 
multiset-trie. Synthetic and real-world data sets are used in experiments to test the performance 
of the data structure. The experiments also highlight the methods for optimizing a multiset-trie.
%
The related work is reviewed in Section~\ref{c:relwork}. 
%
Finally, we conclude with the discussion of future work in Section~\ref{c:conclusions}.
%