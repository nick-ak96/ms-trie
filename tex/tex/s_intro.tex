\section{Introduction} \label{c:introduction}

% multisets def; not ordered
A multiset is a collection of elements that can have more than one instance. As in the case of ordinary sets, the ordering of the elements in multisets is not relevant. For example, multisets $\{1,1,2\}$ and $\{1,2,1\}$ represent the same multiset.

% motivation for the design
Multisets appear in a wide variety of domains and applications \cite{savnik2021plos}. The index structures for storing sets of multisets were studied in the area of object-relational database systems to store, compress and query multiset-valued attributes efficiently \cite{bouros2016set,gripon2012compressing,ross2004symmetric,steinruecken2015compressing}. The need to efficiently manage multisets also appears in the information retrieval \cite{zobel1992efficient,zobel2006inverted,manning2008introduction} where texts are represented as multisets. In data mining, sets are often used to represent and efficiently search hypotheses in the knowledge discovery process \cite{mannila1997,flach1999aicom}. In the area of expert systems, multisets are used for the representation and querying of the preconditions of rules \cite{forgy1982}. Finally, in recent internet applications, efficient representation and search of multisets (such as user requests and object features) have become essential \cite{bayardo2007simlar,xiao2011tods,wang2017vldb} for data cleaning, information integration, community mining, and entity resolution. 

%\medskip\noindent\textbf{Problem definition.}\;
In this paper, we address the problems of storing, indexing, and querying the sets of multisets. In particular, we deal with the design of an index data structure that provides an efficient implementation of the multiset containment queries. Let $S$ be an index storing a set of multisets. For a given input multiset $m$, a \emph{containment query} searches for either sub-multisets or super-multisets of $m$ in $S$. 

Existent indexes for storing a set of multisets are rooted in the search trees \cite{corman2001}. The elements of a search tree can be accessed through keys. This approach is efficient for checking the membership of individual multisets $m$ in $S$. However, it is not as efficient for  containment queries. The search based on the containment relation requires access to the collections $C\subseteq\/S$ of multisets that are related to a multiset $m$ either by a sub-multiset or a super-multiset relationship. 

The existing solutions to the implementation of the containment queries use the inverted indexes \cite{zobel1992efficient,zobel1998inverted,zobel2006csur,broder2006indexing,terrovitis2006cikm,terrovitis2011icdt}, the signature trees \cite{deppisch1986sigir,tousidou2002sigstruc,yangjun2005stree,zobel1998inverted} and, the B+ trees~\cite{Helmer2003,terrovitis2011icdt}. These solutions provide a key-value look-up, where a key is an element of a multiset and a value is a corresponding multiset. The containment operations require multiple key-value index accesses and also additional processing of partial results such as intersection or union of multisets.

%\medskip\noindent\textbf{Proposed data structure.}\;
To improve the efficiency of containment operations, we propose a data structure \emph{multiset-trie} that is designed for storing and processing a finite bounded set of multisets. A multiset-trie generalizes the \emph{set-trie} data structure proposed by Savnik~\cite{savnik2013index,savnik2021plos} that was designed for storage and processing of a finite bounded set of sets. 
The set-trie is an extension of a trie data structure to provide, besides fast search and retrieval of sets, also the efficient implementation of the set containment operations. The set-trie is also a form of the binomial tree \cite{corman2001}.

Multiset-trie provides a space-efficient representation of a set of multisets and efficient multiset containment operations. As in the case of set-trie, the ordering of the elements in multiset-trie is not relevant for the representation of multisets. As a consequence, the efficiency of the multiset containment operations is obtained by selecting the \emph{specific} ordering of multiset elements. This ordering can be exploited for the efficient search in a multiset-trie.  

% intro into mstrie
The multiset-trie is an $n$-ary tree data structure. Each multiset in a multiset-trie is represented by a path from the root to a leaf node. Multiset elements are symbols from an alphabet. Each symbol from the alphabet is represented by a node at a certain level of the multiset-trie. The node stores the multiplicity of an element in a multiset.

A multiset-trie is also a kind of search tree. Similar to a trie, it uses common prefixes for a shared data representation. Unlike the compact prefix tree \cite{Sedgewick:2011:ALG:2011916}, the multiset-trie does not support path compression. However, the absence of path compression makes the multiset-trie a perfectly height-balanced tree. Moreover, when multiset-trie is full, it forms a complete $n$-ary tree.

\medskip\noindent\textbf{Contributions.}\;
The main contributions of this paper are as follows. First, a  multiset-trie is a novel data structure for storing and querying a set of multisets that provides efficient multiset containment operations. 

Second, a mathematical model is developed to analyze the complexity of multiset containment operations. In particular, we estimate the size of a sub-tree traversed by a containment query and give an insight into the time complexity of containment queries. 

In addition to the mathematical analysis, the size of a sub-tree visited by a multiset containment query is also the central focus of the empirical analysis. We carefully designed the experiments to unravel the main features of the search space. We observe how the number of visited nodes depends on various parameters, such as the density of the tree, and the ordering of multiset elements.

Finally, the mathematical, as well as empirical analyses show the influence of ordering the elements of multisets on the efficiency of storing and processing multisets. We show that the ordering, which is based on the frequencies of the multiset elements, can speed up the multiset containment queries by orders of magnitude.

\medskip\noindent\textbf{Paper organization.}\; 
The paper is organized as follows. In the following Section~\ref{c:description}, we present the multiset-trie data structure in detail.
Section~\ref{c:operations} presents the operations of the multiset-trie. These include the basic operations of search trees and multiset containment operations. The algorithms are presented in detail using pseudo code. 

The description of multiset-trie operations is followed by the mathematical analysis of their complexity in Section~\ref{c:analysis}. The main assumption is that multisets are constructed uniformly at random with bounded cardinality. By using probabilistic tools, we describe the time complexity of the algorithms and the space complexity of the structure.

In Section~\ref{c:experiments} we present an empirical study of the multiset-trie. Synthetic and real-world data sets are used in experiments that are designed to study the performance of multiset containment operations. Further, the multiset-trie is empirically compared to the inverted index in a separate experiment. The experiments highlight the methods for optimizing a multiset-trie. 

The related work is reviewed in Section~\ref{c:relwork}. This section describes a set-trie data structure, the inverted indexes, the signature indexes, and the multisets in relational database systems. Finally, the concluding remarks and the future work are presented in Section~\ref{c:conclusions}.
