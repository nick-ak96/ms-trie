\section{Introduction} \label{c:introduction}
%
% general intro into problem
%
A multiset is a data structure that represents a collection of elements. It generalizes a set data structure by allowing duplicate elements in a collection. Multisets appear in a wide variety of domains and applications. The index structures for storing sets of multisets were studied in the area of object-relational database systems to efficiently store, compress and query multiset-valued attributes \cite{bouros2016set,gripon2012compressing,ross2004symmetric,steinruecken2015compressing}. Furthermore, the need to efficiently store and query the multisets appears also in the information retrieval \cite{}, the data mining \cite{}, and in the area of expert systems \cite{}. 
%
% problem

In this paper, we address the problems of storing, indexing and querying the sets of multisets. In particular, we deal with the design of an index data structure that provides an efficient implementation of the containment queries. Let $S$ be an index storing a set of multisets. For a given input multiset $m$, a \emph{containment query} searches for either sub-multisets or super-multisets of $m$ in $S$. 

Existent indexes for storing a set of multisets are rooted in the search trees. The elements of a search tree can be accessed through keys. This approach is efficient for checking the membership of individual multisets $m$ in $S$. However, it is not as efficient for operations that use containment relation between multisets. The search based on the containment relation requires access to the collections $C\subseteq\/S$ of multisets that are related to a multiset $m$ either by a sub-multiset or a super-multisets relationship.

The existing solutions to the implementation of the containment queries include the inverted file \cite{}, signature tree \cite{} or B+ tree~\cite{Helmer2003}. These solutions provide a key-value look-up for elements of a multiset implying that containment operations are still element-wise. (? Is it possible to state more precisely the drawbacks of these solutions? ?)
%

---------------------------------------------------------------------

% description of innovations
%
To improve the efficiency of containment operations, we propose a data structure \emph{multiset-trie} that is designed for storage and processing of a finite bounded set of multisets. It extends the \emph{set-trie} data structure proposed by Savnik~\cite{savnik2013index} that was designed for storage and processing of a finite bounded (?or unbounded?) set of sets. 
%
Set-trie is a trie based data structure that provides a fast search and retrieval of sets and implements set-containment operations. 
%
Multiset-trie generalizes set-trie and can work with a set of sets (as set-trie) or with 
a set of multisets providing efficient multiset-containment operations.

% intro into mstrie
The multiset-trie is an $n$-ary tree based data structure with properties similar to those of a trie. 
Multisets are associated with a collection of nodes in a tree such that every node represents an element 
of a multiset with particular multiplicity bounded by a node degree $n.$
%
Multiset-trie is a kind of search tree. Similarly to a trie, it uses common prefixes for shared data representation.
Unlike the compact prefix tree, the multiset-trie does not support path compression. However, the absence of path 
compression makes the multiset-trie a perfectly height-balanced tree. Moreover, when multiset-trie is full it forms 
a complete $n$-ary tree.
%
The multiset-trie handles multisets directly by having access to each of the element 
without the need to reconstruct them for processing, which allows fast retrieval and 
containment operations. In particular, it supports submultiset and supermultiset queries.
The operations allow to find and retrieve the closest submultisets or supermultisets 
as well as to find and retrieve all of them.

% --------------------------------------------------------------------------------------------------------------------------------------------------------------------

% key results
Empirical studies on real data show, that multiset-trie is sensitive to the context, which can 
be further used to optimize data structure for particular data. Moreover, knowing the data 
it is possible to estimate the overall performance of the multiset-trie both time and space 
related using our mathematical theory that describes multiset-trie.
%
The comparative study shows how efficient multiset-trie is by outperforming inverted file in 
both equality and containment queries by up to 4 orders of magnitude in the time consumed by 
query. 

% --------------------------------------------------------------------------------------------------------------------------------------------------------------------

% paper organization
%
In the following Section~\ref{c:description} we present the organization of 
multiset-trie data structure in detail.
%
Next, in Section~\ref{c:operations} we present operations that multiset-trie currently 
supports. This includes multiset containment and the basic search tree functions. 
The algorithms in pseudo code are presented as well. 
%
The description of multiset-trie functions and procedures is followed by the 
mathematical analysis of their complexity in Section~\ref{c:analysis}. 
The main assumption is that multisets are constructed uniformly at random 
with bounded cardinality. By using probabilistic tools we describe time complexity of 
the algorithms and space complexity of the structure.
%
Further, in Section~\ref{c:experiments} we present an empirical study of the 
multiset-trie. Synthetic and real-world data sets are used in experiments to test the performance 
of the data structure. The experiments also highlight the methods for optimizing a multiset-trie.
%
The related work is reviewed in Section~\ref{c:relwork}. 
%
Finally, we conclude with the discussion of future work in Section~\ref{c:conclusions}.
%