\section{Introduction} \label{c:introduction}
%
% some intro
During recent years the popularity of digital data has increased. Huge amount of diverse data is collected in databases and 
processed regularly. Data processing usually consists of operations such as filtering, sorting and aggregation. This requires 
indexing for optimization of data processing especially when it is repetitive. Data often includes duplicates or multivalued 
attributes that are complicated to process efficiently when sophisticated queries are required.

Multisets or bags are not well supported by storage systems for processing. Usually multisets have to be stored in a 
simpler format that does not allow a direct analysis of the data. Such storage results in a more complex processing, 
since multisets have to be re-constructed or pre-processed first in order to run the query. It also becomes impossible
to execute containment queries on multisets.

Currently the most efficient ways for multisets storage and processing involve inverted files, signature trees and B+ tree.
Such solutions provide key-value look-up only, which does not allow to perform containment queries on multisets directly and efficiently.
Additional processing is required such as intersection or union of obtained partial results from key-value look-ups. The post 
processing of partial results can be optimized, however, it is still not as efficient and usually requires additional encoded 
signatures or other types of identifiers to speed-up the process.

% --------------------------------------------------------------------------------------------------------------------------------------------------------------------

% description of innovations
%
In this paper we propose a data structure \emph{multiset-trie} that is designed for storage and processing of multisets.
It is an extension of the \emph{set-trie} data structure proposed by Savnik \cite{savnik2013index}. 
Set-trie is a trie based data structure that is used for storing and fast retrieval of 
objects represented as sets. The set-trie provides the nearest-neighbor search by 
implementing methods that perform set-containment queries. Multiset-trie generalizes 
set-trie and provides support for storing and retrieving objects that can be represented 
as both sets and multisets. 

The multiset-trie is an $n$-ary tree based data structure with properties similar to those of a trie. 
This particular combination allows us to associate multisets with a collection of nodes in a tree. 
Every node represents an element of a multiset with particular multiplicity. 
%
Multiset-trie is a kind of search tree. Similarly to a trie, it uses common prefixes 
to narrow down the search. It also implements the basic tree methods such as 
search, deletion and insertion. Unlike the compact prefix tree, Patricia, the multiset-trie 
does not provide the ability to compress a path. However, the absence of path 
compression makes the multiset-trie a perfectly height-balanced tree.

The multiset-trie is designed for efficient execution of the multiset containment 
operations. It handles multisets directly by having access to each of the element 
without the need to reconstruct them for processing, which allows fast retrieval and 
containment queries. In particular, it supports submultiset and supermultiset queries.
The operations allow to find and retrieve the closest submultisets or supermultisets 
as well as to find and retrieve all of them.
%
Since multiset is a generalized version of sets that allows duplicates multiset-trie can be 
used for storing and processing sets supporting the same set of operations and providing 
the same efficiency.

% --------------------------------------------------------------------------------------------------------------------------------------------------------------------

% key results
Empirical studies on real data show, that multiset-trie is sensitive to the context, which can 
be further used to optimize data structure for particular data. Moreover, knowing the data 
it is possible to estimate the overall performance of the multiset-trie both time and space 
related using our mathematical theory that describes multiset-trie.
%
The comparative study shows how efficient multiset-trie is by outperforming inverted file in 
both exact and containment queries by up to 4 orders of magnitude in the time consumed by 
query processing. 

% --------------------------------------------------------------------------------------------------------------------------------------------------------------------

% paper organization
%
In the following Section~\ref{c:description} we present the organization of 
multiset-trie data structure. The representation of multisets in multiset-trie is 
explained in detail.
%
In Section~\ref{c:operations} we present operations that multiset-trie currently 
supports. Those are the multiset containment and the basic search tree functions. 
The algorithms in pseudo code are presented as well. 
%
The description of multiset-trie functions and procedures is followed by the 
mathematical analysis of their complexity in Section~\ref{c:analysis}. 
We assume that multisets are constructed uniformly at random and are parametrized 
by several parameters, such as multiplicity and the alphabet. By using probabilistic 
tools we describe time complexity of the algorithms and space complexity of the structure.
%
Further, in Section~\ref{c:experiments} we present an empirical study of the 
multiset-trie. Artificially generated as well as real-world data sets are used in experiments. 
The experiments are dedicated to testing the performance of the data structure while 
varying selected parameters. The experiments also highlight the methods for optimizing a multiset-trie.
%
The Section~\ref{c:relwork} presents related work. The connection to the set-trie 
data structure \cite{savnik2013index} is discussed more explicitly. We also relate 
the multiset-trie to the information retrieval systems. In particular, we refer to the 
inverted index data structure and discuss how the multiset-trie can be used as a 
database index.
%
Finally, in Section~\ref{c:conclusions} our conclusion about the multiset-trie 
data structure and discussion of future work are presented.